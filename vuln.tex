\section{Vulnerabilities in embedded interpreters}
\label{s:vuln}

\begin{figure*}
\small
\centering
\begin{tabular}{llll}
\toprule
Vulnerability type & Vulnerabilities & Virtual machine & Software
\\ \midrule
Resource exhaustion
&                    CVE-2010-2286 & UDVM       & Wireshark \\
&                    CVE-2010-3880 & \inetdiag & Linux \\
&                    CVE-2011-2213 & \inetdiag & Linux \\
&                    CVE-2011-3627 & LLVM       & ClamAV \\
Arithmetic errors
&                    CVE-2010-5137 & Bitcoin    & bitcoind \\
&                    CVE-2007-3725 & RarVM      & ClamAV \\
Information leak
&                    CVE-2010-4158 & BPF        & Linux \\
&                    CVE-2012-3729 & BPF        & Apple iOS \\
Arbitrary code execution
&                    CVE-2010-4347 & AML        & Linux \\
&                    CVE-2011-1021 & AML        & Linux \\
%&                    CVE-2007-5741 & Pickle     & Plone CMS \\
&                    CVE-2011-2520 & Pickle     & Fedora firewall config \\
&                    CVE-2012-4406 & Pickle     & Openstack-Swift \\
Memory corruption
&                    CVE-2010-2995 & UDVM       & Wireshark \\
%&                    CVE-2010-2287 & UDVM       & Wireshark \\
%&                    CVE-2009-3743 & TrueType   & Ghostscript \\
%&                    CVE-2010-1797 & TrueType   & Apple iOS \\
%&                    CVE-2010-3814 & TrueType   & FreeType \\
&                    CVE-2010-2520 & TrueType   & FreeType \\
&                    VU\#662243    & RarVM      & Sophos Antivirus \\
\bottomrule
\end{tabular}
\caption{Examples of vulnerabilities of embedded interpreters.}
\label{fig:vuln}
\end{figure*}

%% CVE-2011-2520: privilege escalation (system-config-firewall)


An embedded interpreter that receives code from an untrusted source must validate
it before execution.  In addition, if input data to the bytecode is also
untrusted, the interpreter must execute the bytecode defensively.
This section illustrates common vulnerabilities found in embedded interpreters
that fail to do so. \autoref{fig:vuln} summaries the
vulnerabilities studied in this section.

\subsection{Resource exhaustion}
\label{s:vuln:loop}

When an embedded interpreter runs bytecode,
it should be able to control resource consumption.
If the bytecode supports jumps~\cite{aml:spec,rar:vm,clamav:llvm},
it is possible to contain backward jumps that will lead to
infinite loops;
the interpreter should be able to constrain such jumps.
Similarly, if the bytecode supports subroutines,
% or runs on a stack machine,
the interpreter should guard against infinite recursion that will lead
to stack overflows.

%Infinite loops may happen to even in VMs that by
%design do not allow loops.
\autoref{fig:2213} shows a vulnerable code snippet, from the
\inetdiag interpreter in the Linux kernel,
as well as the corresponding patch.  The interpreter
increases the instruction pointer by \cc{inet\_diag\_bc\_op->yes} when the
current instruction evaluates to true.
%
To forbid backward jumps, the code defines ``\cc{yes}'' as an
unsigned integer; but it forgets to validate that \cc{yes}
is non-zero, leaving the interpreter vulnerable to infinite loops.

A general way to constrain infinite loops or infinite recursion is to limit the
number of instructions executed and the depth of nested recursion.  Every time an
instruction is executed or the recursion depth is increased, the counter is
incremented. The interpreter aborts the execution when the counter reaches the
limit. However, implementing the watchdog counter is non-trivial in complex VMs,
as we can find in CVE-2010-2286 and CVE-2011-3627. Without careful plans for
restricting resources, it is hard to avoid DoS in embedded interpreters.

%% Complex interpreter logic, however, makes managing the counter
%% difficult. Missing one increment could expose the bytecode VM 
%% to infinite loops as in CVE-2010-2286 and CVE-2011-3627.
%% \XXX[yd][This paragraph does not make much point. The two sentence are
%% not well related.]

\begin{figure}
\centering
\input{code/inetdiag-loop}
\coderule
\caption{The patch that fixes CVE-2011-2213.  The vulnerability was missing
validation of \inetdiag instructions in the Linux kernel, which results in
infinite loop if \cc{op->yes} is zero.}
\label{fig:2213}
\end{figure}

%stack: CVE-2011-3627 (\clamav)

\subsection{Arithmetic errors}
\label{s:vuln:arith}

Embedded interpreters
that provide arithmetic instructions are prone to arithmetic errors.
These errors may come from
either the incompatible formatting between bytecode and machine instructions,
or unexpected behaviors of a host machine in arithmetic.  For example, when
handing a signed division instruction, an interpreter
should check that the divisor is
non-zero, and that the quotient does not overflow (i.e., not $\cc{INT_MIN} /
-1$); otherwise, the division might trigger machine exception probably resulting
in program termination.

Worse yet, checking for overflow condition
turns out to be tricky and error-prone.
\autoref{fig:clamav} shows the ClamAV interpreter attempting to
verify division operands but actually mixing up the dividend and
divisor, allowing a denial-of-service attack.
Similarly, argument checks to ``fix'' the DragonFlyBSD
bug~\#1748~\cite{dbsd:div} were incorrect, 
enabling specially crafted BPF filters to cause a kernel panic.

Bit shifting also leads to arithmetic errors.  For example, CVE-2010-5137 shows
that a missing check of large bit shifts causes Bitcoin to crash on some
machines when processing a transaction containing a left-shift opcode.

%div by 0: \cc{http://bugs.dragonflybsd.org/issues/1748}, BPF in NetBSD and DragonFlyBSD).

%sdiv INT_MIN/-1: ClamAV~\cite{clamav:sdiv}.

\begin{figure}
\centering
\input{code/clamav-sdiv}
\coderule
\caption{Incorrect handling of the signed division opcode in \clamav's
interpreter.  The correct check to avoid signed division overflow should swap
\cc{a} and \cc{b}, i.e. \cc{(a == INT64\_MIN \&\& b == -1)}.}
\label{fig:clamav}
\end{figure}

%shift: CVE-2010-5137 (bitcoin)

\subsection{Information leak}
\label{s:vuln:leak}

When VMs provide information to the bytecode or the bytecode can access outside
of VM scope, they can reveal information to the outside of VMs. In practice, an
information leak often happen when a VM allocates memory without initialization
and provides the allocated buffer to the bytecode to run. This type of
vulnerability can be seen when a VM simulates register files or uses stacks for
execution. Malicious bytecode can output those values and obtain sensitive
information from the host system.
 
\autoref{fig:bpf} shows the BPF interpreter in the OpenBSD kernel, which does
not zero out the \cc{mem} array on the kernel stack. A malicious filter might
return the value of an unused memory slot as the number of accepted bytes,
potentially exposing kernel randomness to the user space.  A similar
vulnerability (CVE-2012-3729) was discovered in the Linux kernel too.

\begin{figure}
\centering
\input{code/bpf-leak}
\coderule
\caption{
CVE-2012-3729:
OpenBSD BPF filters loading from uninitialized scratch memory 
(\cc{mem} array) could leak sensitive information from the kernel stack.
The patch initializes it with \cc{bzero}. }
\label{fig:bpf}
\end{figure}

\subsection{Arbitrary code execution}
\label{s:vuln:exec}

Enabling external calls breaks the isolation between the interpreter and the
host system. This lacks of isolation often allows bytecode to execute arbitrary
code in the host system.  This is a dangerous VM design that should be avoided,
unless both the bytecode and inputs come from trusted sources.

Linux's AML VM is able to invoke any kernel functions. Usually this is not a
problem because the AML bytecode comes from trusted hardware. However, a kernel
developer accidentally exposed the custom control method interface to
unauthorized users, leading to a severe vulnerability (CVE-2010-4347).

Another example is the notorious Pickle module in Python.  The
Pickle VM defines \cc{GLOBAL} and \cc{REDUCE} opcodes that
provide the capability for importing and running arbitrary Python
functions. Many Python developers are unaware of this security
flaw and use pickle to deserialize objects from untrusted sources,
permitting remote code execution~\cite{pickle:wp}, such as in 
CVE-2012-4406 and CVE-2011-2520.


\subsection{Memory corruption}

Many embedded interpreters are written in an unsafe language, and programming errors can
lead to memory corruption errors. Examples include mishandling of the execution
stack (\autoref{fig:freetype}), omitting bound checks when executing string
operations, etc.  We do not dive further into these problems since they are the
result of implementation errors, as opposed to design issues specific to
embedded interpreters.

\begin{figure}
\centering
\input{code/freetype-ovf}
\coderule
\caption{
CVE-2010-1797:
Stack-based buffer overflow in the FreeType. 
The Type 2 Charstring VM forgets to
validate the stack pointer after processing an operator. }
\label{fig:freetype}
\end{figure}


\subsection{JIT spraying}

Some embedded interpreters~\cite{mccanne:bpf,clamav:llvm} allow to
compile their bytecode to machine instructions for better performance.
%Although we could not find a reported vulnerability
%directly originated from a JIT implementation,
JITted bytecode introduces new attack vectors to the host system.
For example, attackers
can place a shell code to the kernel by writing a benign-looking BPF
bytecode. When it is compiled and loaded into the kernel, attackers can trigger
another bug to jump into the middle of the compiled code, so that the host
machine interprets the instructions as the attackers intended. This attack
technique is called JIT spraying and it can be used to break the hardened x86
Linux with Supervisor Mode Execution Protection~(SMEP) that is supposed to
prevent OS from executing user-level code
inadvertently~\cite{mcaalister:jit}. Unlike well-aligned bytecode, instructions
of host machines vary in size and misaligned so it is hard to avoid extra attack
vectors when enabling JIT, without careful verification of the
compiled machine code.
